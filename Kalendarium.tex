% !TEX TS-program = LuaLaTeX+se %% required to run gregoriotex. XeLaTeX could be used with fontspec if gregoriocolor is removed. 
\documentclass[11pt]{book} %% twoside is default
%%%%%%%%%%%%%%% STANDARD PACKAGES %%%%%%%%%%%%%%%

%% for personal reasons, the commands are a mix of French and English names. Feel free to choose one or the other if you make use of them. This file is subject to change later on.

%% This is the format of the recent Solesmes books.
%\usepackage[paperwidth=6in, paperheight=9in]{geometry}
\usepackage[paperheight=205mm,paperwidth=135mm]{geometry}
\usepackage{fontspec}
\usepackage[ecclesiasticlatin.usej]{babel}
     \babelprovide[hyphenrules=latin]{ecclesiasticlatin}
          \usepackage{xspace}
%     \usepackage{multicol}
%\usepackage{paracol}
%\footnotelayout{m} %% this allows merged footnote if text is in parallel columns (mostly translations)
\usepackage{fancyhdr}
\usepackage{needspace} %%pour réserver de l'espace en bas de page ; ça évite des séparations entre les titres et la partition ou le texte qui les suivent.
\usepackage[compact]{titlesec}
\usepackage{xcolor}
%\usepackage{xstring}
\usepackage{enumitem} %% nécessaire pour les textes des psaumes.
\usepackage{hyperref}
\usepackage{refcount}


%%%%%%%%%%%%%%% HYPHENATION AND TYPOGRAPHICAL CONVENTIONS %%%%%%%%%%%%%%

%% page settings %%%%
\geometry{bindingoffset=5mm,
inner=10mm,
outer=10mm,
top=12mm,
bottom=15mm,
headsep=3mm,
} %%borrowed from Matthias, based on the Solesmes books; binding offset TBD.

\AtBeginDocument{\setlength{\parindent}{1em}} %% should set 1em to EB Garamond not Computer Modern.

%%%%%%%%%%%%%%% GREGORIO CONFIG %%%%%%%%%%%%%%%

\usepackage[autocompile]{gregoriotex}

%% sets * and † to font called via fontspec
\def\GreStar{*}
\def\GreDagger{†}
%\gresetspecial{cross}{\grecross} %% to use cross from Gregorio fonts which is thin and doesn't especially match text well.

%produces a score with a smaller initial (default is 40 pt) and annotations like in the Solesmes books; the 1st is optional and can be omitted as needed.
 \NewDocumentCommand{\gscore}{O{}mm}{%
 \greannotation{#1}%
\greannotation{#2}%
\grechangestyle{initial}{\fontsize{28}{28}\selectfont}%
 \gregorioscore{partitions/#3}}

%% score with no initial, e.g psalms, common tones, etc.
\newcommand{\smallscore}[2][y]{
  \gresetinitiallines{0}
  \gregorioscore{partitions/#2} %% à remplacer avec NewDocumentCommand ; les arguments ne marchent pas correctement … que fait le [y] ?
  \gresetinitiallines{1}
}

 \NewDocumentCommand{\MagAnnotation}{}{%
 \grechangedim{annotationseparation}{-0.01cm}{fixed}%
 } %% pour les antiennes où le chiffre convient mieux aux miniscules (parfois 1, 2, 4, 7…) ; il n'est pas nécessaire pour 6 ou 8.
%\gresetheadercapture{annotation}{greannotation}{string}

%% text above lines shall be italicized and in a smaller font
\grechangestyle{abovelinestext}{\it\fontsize{8}{10}\selectfont}

%% We do not want alt (A-bove l-ines t-ext) to be it for psalm and canticle endings,, however;  Roman/upright text is what the Liber Usualis uses.

%% fine-tuning of space beween the staff and the text above lines (used for Magnificats; needs to be rethought (again, should 2 and 3 be raised or not?) and worked into \smallscore
\newcommand{\altraise}{-0.2cm} %% default is -0.1cm %% needs to be fine-tuned for \rubrique command with EB Garamond font. -0.2cm is MB value
\grechangedim{abovelinestextraise}{\altraise}{scalable}

\newcommand{\altheight}{0.5cm} %% default is 0.3cm and value must be bigger than text height; this should fix the problem. but needs further investigation.
\grechangedim{abovelinestextheight}{\altheight}{scalable} %% this is a very finicky command.

%% allows printing of glyphs from Gregorio score font (available glyphs listed in documentation) as text.
\makeatletter
\def\gretextglyph#1{{\gre@font@music\csname GreCP#1\endcsname}}
\makeatother

%%% Font %%%
\setmainfont{EB Garamond}[UprightFont=EB Garamond Regular,
ItalicFont= EB Garamond Italic,
BoldFont= EB Garamond Bold,
Ligatures=Rare,
Numbers=Proportional,
Numbers=OldStyle] %% all \kern numbers are based on this and can be removed if this font is abandoned.
\newfontfamily\symbolfont{liturgy}
  %% text must be written {\symbolfont{+}} (you can use V and R as well) but may conflict as + is † in gabc. Cross should be \small or even \footnotesize
%\newfontfamily\symbolfont2{BorgiaPro-Regular}

%%%% Font %%%
%\setmainfont{EB Garamond}[UprightFont=EB Garamond Regular,
%ItalicFont= EB Garamond Italic,
%BoldFont= EB Garamond Bold,
%Ligatures=Rare,
%StylisticSet=6,
%Numbers=Proportional,
%Numbers=OldStyle] %% all \kern numbers are based on this and can be removed if this font is abandoned. %%StylisticSet=6 is, in Pardo EBGaramond, the long-tailed Q.
%\newfontfamily\symbolfont{liturgy} 
  %% text must be written {\symbolfont{+}} (you can use V and R as well) but may conflict as + is † in gabc. Cross should be \small or even \footnotesize


%%% Typographical Commands %%%%

%%rubrics: black italics, smaller than body of psalms etc

\NewDocumentCommand{\rubrique}{m}{{\fontsize{8}{10}\selectfont\textit{#1}}}

%\NewDocumentCommand{\rubrique}{m}{{\scriptsize\textit{#1}}}

%% macro to print Alleluia for versicles.
\NewDocumentCommand{\tpalleluia}{}{(\textit{T.P.} Allelúia.)}

%% macro to print normal text inside of rubric (name of a chant or prayer, etc.)

\NewDocumentCommand{\normaltext}{m}{{\normalfont\fontsize{8}{10}\selectfont{#1}}}

%% in case something should be bolded inside of a rubrique
\NewDocumentCommand{\rubriquegras}{m}{{\fontsize{8}{10}\selectfont\textbf{#1}}}

%% to print in red instead of italicizing
\NewDocumentCommand{\rouge}{m}{\textcolor{gregoriocolor}{\fontsize{8}{10}\selectfont{#1}}}

%% to print in black within rouge group.
\NewDocumentCommand{\textenoir}{m}{\textcolor{black}{\fontsize{8}{10}\selectfont\normalfont{#1}}}

%% rouge but in italic (this is technically not correct).
\NewDocumentCommand{\rougeit}{m}{\textcolor{gregoriocolor}{\fontsize{8}{10}\selectfont\textit{#1}}}

%\newcommand{\rubriquegras}[1]{{\fontsize{9}{11}\selectfont\textbf{#1}}}

%%macro to space punctuation with ecclesiasticlatin language from babel.
\NewDocumentCommand{\espaceponctuation}{}{\hspace{0.10em}} %%this value seems more balanced with this font than the definition provided in the ecclesiasticlatin documentation.

\NewDocumentCommand{\myrule}{}{%
    \par%
    {%
        \centering%
        \rule{0.3\textwidth}{0.4pt}%
        \par%
    }% typesets a horizontal rule like on the page with the prayers before and after the office.
}  %%If you're using color at all in your document, you might want to either force \myrule to use black or make it cusotmizable. (from u/Independent-Comb-257)


%%% typesets a cross pattée.

\NewDocumentCommand{\mycross}{}{%
{\symbolfont\footnotesize{{+}}}%
{}
} %% can replace <+> with <U+2720> if a suitable font is found (or EB Garamond font is fixed); command will be useful in lieu of typing the Unicode.

%% macro to format psalm text

\NewDocumentCommand{\pstexte}{ m }{%
    \smallskip%
    \noindent%
    \begin{itemize}[%
    		label=\null, %
			leftmargin=0pt, %
			itemindent=0mm, %
			labelsep=0pt, %
			labelwidth=0pt, %
			rightmargin=0pt, %
			parsep=0pt, %
			topsep=0pt, %
			itemsep=0pt]%
    \input{psaumes/#1}
    \end{itemize}}
    
    %% Command de Matthias Bry modifié, prints psalm incipit score with the text
    \NewDocumentCommand{\psalmus}{mmm}{
	\needspace{4\baselineskip}
	\smalltitle{Psalmus #1.}
	\smallscore[n]{#2}
	\pstexte{#3}
}
    
    %% macro to print any additional text (Capitiulum, oratio, rubrics)
 \NewDocumentCommand{\textes}{ m }{%
    \input{textes/#1}}
    
    
    %% SC work for page headers and for secondary headings but not so much for things like "Dominica…" and certainly not the feast name%%
    
    %%%% Headers%%%%
    \pagestyle{fancy}
\fancyhead{}
\fancyfoot{}
\renewcommand{\headrulewidth}{0pt}


%\fancyhead[RO]{\small\thepage}
%\fancyhead[LE]{\small\thepage}

\fancyhead[RO]{\small\rightmark\hspace{0.5cm}\thepage}
\fancyhead[LE]{\small\thepage\hspace{0.5cm}\leftmark}

\newcommand{\setheaders}[2]{
	\renewcommand{\rightmark}{{\sc#2}}
	\renewcommand{\leftmark}{{\sc#1}}
}
\setheaders{}{}


%% TITLE Commands %%%%
\titleformat{\section}[display]{\large\filcenter\sc\addfontfeature{LetterSpace=3.0}}{}{}{}
\titleformat{\subsection}[display]{\large\filcenter\sc\addfontfeature{LetterSpace=3.0}}{}{}{}
\setcounter{secnumdepth}{0}
\titlespacing{\section}{0pt}{*0}{*0}

%% all of these are a mess and should probably be ignored, but feel free to use them by referring to Matthias B's originals in the Nocturnale Romanum repository.

\newcommand{\officiumtitulum}[1]{
%  \newpage
%  \thispagestyle{empty}
  \setheaders{{\scshape\addfontfeature{LetterSpace=3.0}#1}}{{\scshape\addfontfeature{LetterSpace=3.0} #1}}
  \begin{center}
  {\scshape\large #1}\par
  \end{center}}
  
%   \setheaders{{{\scshape\addfontfeature{LetterSpace=3.0}}}{{{\scshape\addfontfeature{LetterSpace=3.0} {#1}}}
\NewDocumentCommand{\smalltitle}{m}{
\needspace{5\baselineskip}  %% very small if there are neumes above the staff, including flats in mode 2, e.g. O Doctor optime
\vspace{\baselineskip}
 {\centering #1\par}
}

\newcommand{\subtitulum}[1]{
 \subsection{
  \begin{center}
  {\large#1}
  \end{center}
}}

%% originally used apparently for typesetting portions of the Common of the BVM
\newcommand{\espacetitre}{\vspace{-0.5cm}}
\newcommand{\espaceps}{\vspace{-3mm}} %% ps =psaume
\newcommand{\espacecap}{\vspace{-3mm}} %% cap is capitulum
\newcommand{\espace}{\vspace{2mm}}


 %% based on file of same name from Nocturnale Romanum project with additional commands (mostly to change spacing, simplify score input or reduce copy-and-paste

\raggedbottom

\usepackage{ragged2e}
\usepackage{geometry} 
\usepackage[parfill]{parskip}

\usepackage{longtable}
\usepackage{multirow,makecell}

\usepackage{colortbl}

\usepackage{layout}
\pagestyle{plain}
\setlength{\columnsep}{4mm}
\setlength{\parindent}{0mm}
\setlength{\marginparwidth}{7mm}
\setlength{\marginparsep}{3mm}
\setlength{\headsep}{10pt}

\usepackage{microtype}
\usepackage[defaultlines=2,all]{nowidow}
\usepackage[hyphenation,lastparline,nosingleletter]{impnattypo}
\usepackage{epsfig}
\spaceskip=1.0\fontdimen2\font plus 3\fontdimen3\font minus 0.8\fontdimen4\font %% this probably needs adjusting as I don't use the same fontss

%% these commands use Matthias B's commands but in a cleaner and shorter format. Could be moved to commonheaders.

\NewDocumentCommand\hang{}{\setlength{\hangindent}{10pt}}
\NewDocumentCommand\mem{m}{\textit{Comm.} #1}
\NewDocumentCommand\gcolor{m}{\textcolor{gregoriocolor}{#1}} %% requires gregoriotex (declared in commonheaders). Months should be in black italic type if not using any red. Dominical Letters can remain in roman type.

\NewDocumentCommand\capspace{m}{{\addfontfeature{LetterSpace=5.0}{#1}}}

\NewDocumentCommand\scspace{m}{\textsc{{{\addfontfeature{LetterSpace=5.0}{#1}}}}}

\begin{document}

% !TEX TS-program = LuaLaTeX+se
% !TEX root = Kalendarium.tex

\lineskiplimit = 0pt
\markboth{}{}
\thispagestyle{empty}
{\centering {\large \capspace{TABELLA FESTORUM MOBILIUM}.\par}}\label{ephemeridae} \par %% will need to fix label once I learn how to do that… %% need to adjust spacing
\setlength\LTleft{0pt}
\setlength\LTright{0pt}
\setlength{\tabcolsep}{0pt}
\renewcommand{\arraystretch}{1.25}
\newcommand{\boldhline}{\noalign{\global\arrayrulewidth1.5pt}\hline\noalign{\global\arrayrulewidth1pt}}
\newcommand{\thinhline}{\noalign{\global\arrayrulewidth0.5pt}\hline\noalign{\global\arrayrulewidth1pt}}
\newcommand{\whiteline}{\noalign{\global\arrayrulewidth4pt}\hline\noalign{\global\arrayrulewidth1pt}}
\newcommand{\STAB}[1]{\begin{tabular}{@{}c@{}}#1\end{tabular}}
\fontsize{7.5}{7.5}\selectfont
\begin{longtable}{|>{\centering}m{0.12\textwidth}|>{\centering}m{0.08\textwidth}|>{\centering}m{0.15\textwidth}|>{\centering}m{0.15\textwidth}|>{\centering}m{0.16\textwidth}|>{\centering}m{0.16\textwidth}|>{\centering\arraybackslash}m{0.15\textwidth}|}
\arrayrulecolor{black}\boldhline
\gcolor{Anno} & \gcolor{Litt.\\dom.} & \gcolor{Feria IV\\Cinerum} & \gcolor{Pascha} & \gcolor{Ascensio} & \gcolor{Pentecostes} & \gcolor{Adventus} \\
\boldhline
\endfirsthead
\endhead
\endfoot
\endlastfoot
\arrayrulecolor{black} 
\rule{0pt}{3.5mm}2022 & 	\rule{0pt}{3.5mm}b & 	\rule{0pt}{3.5mm}2 mar. &		\rule{0pt}{3.5mm}17 apr. & 	\rule{0pt}{3.5mm}26 maii & 		\rule{0pt}{3.5mm}5 junii & 		\rule{0pt}{3.5mm}27 nov.\\
2023 & 				\gcolor{A} & 		22 febr. & 					9 apr. & 					18 maii & 					28 maii &					3 dec.\\
2024 & 				g f & 					14 febr. & 					31 mar. & 					9 maii & 					19 maii & 					1 dec.\\
2025 &				e & 					5 mar. & 					20 apr. & 					29 maii & 					8 junii & 					30 nov.\\[0.5mm]
\arrayrulecolor{gregoriocolor} \thinhline \arrayrulecolor{black}
\rule{0pt}{3.5mm}2026 & 	\rule{0pt}{3.5mm}d & 	\rule{0pt}{3.5mm}18 febr. & 	\rule{0pt}{3.5mm}5 apr. & 		\rule{0pt}{3.5mm}14 maii & 		\rule{0pt}{3.5mm}24 maii & 		\rule{0pt}{3.5mm}29 nov.\\
2027 &				c & 					10 febr. & 					28 mar. & 					6 maii & 					16 maii & 					28 nov.\\
2028 &				b \gcolor{A} & 	1 mar. & 					16 apr. & 					25 maii & 					4 junii & 					3 dec.\\
2029 &				g & 					14 febr. & 					1 apr. & 					10 maii & 					20 maii &					2 dec.\\[0.5mm]
\arrayrulecolor{gregoriocolor} \thinhline \arrayrulecolor{black}
\rule{0pt}{3.5mm}2030 & 	\rule{0pt}{3.5mm}f & 		\rule{0pt}{3.5mm}6 mar. & 		\rule{0pt}{3.5mm}21 apr. & 	\rule{0pt}{3.5mm}30 maii & 		\rule{0pt}{3.5mm}9 junii & 		\rule{0pt}{3.5mm}1 dec.\\
2031 &				e & 					26 febr. & 					13 apr. 					& 22 maii 					& 1 junii & 					30 nov.\\
2032 &				d c & 				11 febr. & 					28 mar. 					& 6 maii 					& 16 maii & 				28 nov.\\
2033 &				b & 					2 mar. & 					17 apr. 					& 26 maii 					& 5 junii & 					27 nov.\\[0.5mm]
\arrayrulecolor{gregoriocolor} \thinhline \arrayrulecolor{black}
\rule{0pt}{3.5mm}2034 & 	\rule{0pt}{3.5mm}\gcolor{A} & \rule{0pt}{3.5mm}22 febr. & \rule{0pt}{3.5mm}9 apr. & \rule{0pt}{3.5mm}18 maii & 		\rule{0pt}{3.5mm}28 maii & 		\rule{0pt}{3.5mm}3 dec.\\
2035 &				g & 					7 febr. & 					25 mar. & 					3 maii & 					13 maii  & 					2 dec.\\
2036 &				f e & 					27 febr. & 					13 apr. & 					22 maii & 					1 junii & 					30 nov.\\
2037 &				d & 					18 febr. & 					5 apr. & 					14 maii & 					24 maii & 					29 nov.\\[0.5mm]
\arrayrulecolor{gregoriocolor} \thinhline \arrayrulecolor{black}
\rule{0pt}{3.5mm}2038 & 	\rule{0pt}{3.5mm}c & 	\rule{0pt}{3.5mm}10 mar. & 	\rule{0pt}{3.5mm}25 apr. & 	\rule{0pt}{3.5mm}3 junii & 		\rule{0pt}{3.5mm}13 junii & 		\rule{0pt}{3.5mm}28 nov.\\ 		
2039 &				b & 					23 febr. & 					10 apr. & 					19 maii & 					29 maii  & 					27 nov.\\
2040 &				\gcolor{A} g & 	15 febr. & 					1 apr. & 					10 maii & 					20 maii  & 					2 dec.\\
2041 &				f & 					6 mar. & 					21 apr. & 					30 maii & 					9 junii  & 					1 dec.\\[0.5mm]
\arrayrulecolor{gregoriocolor} \thinhline \arrayrulecolor{black}
\rule{0pt}{3.5mm}2042 & 	\rule{0pt}{3.5mm}e & 	\rule{0pt}{3.5mm}19 febr. & 	\rule{0pt}{3.5mm}6 apr. & 		\rule{0pt}{3.5mm}15 maii & 		\rule{0pt}{3.5mm}25 maii  & 	\rule{0pt}{3.5mm}30 nov.\\
2043 &				d & 					11 febr. & 					29 mar. & 					7 maii & 					17 maii  & 					29 nov.\\
2044 &				c b & 				2 mar. & 					17 apr. & 					26 maii & 					5 junii & 					27 nov.\\
2045 &				\gcolor{A} & 		22 febr. & 					9 apr. & 					18 maii & 					28 maii & 					3 dec.\\[0.5mm]
\arrayrulecolor{gregoriocolor} \thinhline \arrayrulecolor{black}
\rule{0pt}{3.5mm}2046 & 	\rule{0pt}{3.5mm}g & 	\rule{0pt}{3.5mm}7 febr. & 		\rule{0pt}{3.5mm}25 mar. & 	\rule{0pt}{3.5mm}3 maii &		\rule{0pt}{3.5mm}13 maii & 		\rule{0pt}{3.5mm}2 dec.\\
2047 &				f & 					27 febr. & 					14 apr. & 					23 maii & 					2 junii  & 					1 dec.\\
2048 &				e d & 				19 febr. & 					5 apr. & 					14 maii & 					24 maii & 					29 nov.\\
2049 &				c & 					3 mar. & 					18 apr. & 					27 maii & 					6 junii & 					28 nov.\\[0.5mm]
\arrayrulecolor{gregoriocolor} \thinhline \arrayrulecolor{black}
\rule{0pt}{3.5mm}2050 & 	\rule{0pt}{3.5mm}b & 	\rule{0pt}{3.5mm}23 febr. & 	\rule{0pt}{3.5mm}10 apr. & 	\rule{0pt}{3.5mm}19 maii & 		\rule{0pt}{3.5mm}29 maii & 		\rule{0pt}{3.5mm}27 nov.\\
2051 &				\gcolor{A} & 		15 febr. & 					2 apr. & 					11 maii & 					21 maii & 					3 dec.\\
2052 &				g f & 					6 mar. & 					21 apr. & 					30 maii & 					9 junii & 					1 dec.\\
2053 &				e & 					19 febr. & 					6 apr. & 					15 maii & 					25 maii & 					30 nov.\\[0.5mm]
\arrayrulecolor{gregoriocolor} \thinhline \arrayrulecolor{black}
\rule{0pt}{3.5mm}2054 & 	\rule{0pt}{3.5mm}d & 	\rule{0pt}{3.5mm}11 febr. & 	\rule{0pt}{3.5mm}29 mar. & 	\rule{0pt}{3.5mm}7 maii & 		\rule{0pt}{3.5mm}17 maii & 		\rule{0pt}{3.5mm}29 nov.\\
2055 &				c & 					3 mar. & 					18 apr. & 					27 maii & 					6 junii & 					28 nov.\\
2056 &				b \gcolor{A} & 	16 febr. & 					2 apr. & 					11 maii & 					21 maii & 					3 dec.\\
2057 &				g & 					7 mar. & 					22 apr. & 					31 maii & 					10 junii & 					2 dec.\\[0.5mm]
\arrayrulecolor{gregoriocolor} \thinhline \arrayrulecolor{black}
\rule{0pt}{3.5mm}2058 & 	\rule{0pt}{3.5mm}f & 		\rule{0pt}{3.5mm}27 febr. & 	\rule{0pt}{3.5mm}14 apr. & 	\rule{0pt}{3.5mm}23 maii & 		\rule{0pt}{3.5mm}2 junii & 		\rule{0pt}{3.5mm}1 dec.\\
2059 &				e & 					12 febr. & 					30 mar. & 					8 maii & 					18 maii & 				30 nov.\\
2060 &				d c & 				3 mar. & 					18 apr. & 					27 maii & 					6 junii & 					28 nov.\\
2061 &				b & 					23 febr. & 					10 apr. & 					19 maii & 					29 maii & 					27 nov.\\[0.5mm]
\thinhline
\end{longtable}

\pagebreak

\normalsize

% TABLE DU CALENDRIER

%%% modeled on the table in the Liber antiphonarius 1949.

%% the original syntax for centering has been fixed. This now puts the leap-year rubric in the right place on the page. Centering each table might work if  vertical rules should be aligned (the second table on each page isn't aligned)

{\centering{\large \capspace{KALENDARIUM PERPETUUM}.}\par} \thispagestyle{empty} \markboth{Calendrier}{Calendrier} \nopagebreak \par \nopagebreak\vspace{5mm}\label{calendrier} %%will need to fix the label later
\setlength\LTleft{0pt}
\setlength\LTright{0pt}
\setlength{\tabcolsep}{5pt}
\renewcommand{\arraystretch}{1.4}
\fontsize{8}{9}\selectfont
%\begin{longtable}{>{\centering}p{0.025\textwidth}|>{\raggedleft}p{0.025\textwidth}|>{\raggedright\arraybackslash}p{0.85\textwidth}}
%\boldhline
%\multirow{1.5}{*}{\STAB{\rotatebox[origin=c]{90}{{\footnotesize \gcolor{L.D.}}}}} & \multirow{1.5}{*}{\STAB{\rotatebox[origin=c]{90}{{\footnotesize \gcolor{Jour}}}}} &  \multicolumn{1}{c}{\multirow{1.75}{*}{{\footnotesize \gcolor{Mois}}}} \\[8.5pt]
%\boldhline
%\null & \null & \null\\[2pt]
%\endfirsthead
%\boldhline
%\multirow{1.5}{*}{\STAB{\rotatebox[origin=c]{90}{{\footnotesize \gcolor{L.D.}}}}} & \multirow{1.5}{*}{\STAB{\rotatebox[origin=c]{90}{{\footnotesize \gcolor{Jour}}}}} &  \multicolumn{1}{c}{\multirow{1.75}{*}{{\footnotesize \gcolor{Mois}}}} \\[8.5pt]
%\boldhline
%\null & \null & \null\\[2pt]
%\endhead
%\endfoot
%\endlastfoot
%% this commented-out code originated from Matthias B's version, but there is no need for headers per the Latin original.

% DEBUT CALENDRIER
%\null & \null & \null\\[1pt]

%% longtable repeats in each month file. This means the settings need to be changed each time unless there is a cleaner way to do this.

%% Currently scshape or capshape is repeated after the abbreviation when S., B. or SS. is used in the title of a feast. This avoids awkward spacing of periods, which fall outside the braces.

%% Line breaks after Comm. are OK per the Liber antiphonarius.

{\centering{\normalsize \gcolor{Januarius.}}\par}

\begin{longtable}{>{\centering}p{0.025\textwidth}|>{\raggedright}p{0.040\textwidth}|>{\raggedleft}p{0.025\textwidth}|>{\raggedright\arraybackslash}p{0.85\textwidth}}
%\begin{longtable}{>{\centering}p{0.025\textwidth}|>{\raggedleft}p{0.025\textwidth}|>{\raggedright\arraybackslash}p{0.85\textwidth}}
\gcolor{A} &Kal. & 1 & \hang \scspace{Circumcisio Domini} et Octava Nativitatis. \gcolor{Duplex II classis.}\\
\null & \null & \null &  \hang \textit{Dominica inter Circumcisionem et Epiphaniam.}  \scspace{Ss. Nominis Jesu}.  \gcolor{Duplex II classis.}\\
b &iv & 2 & \hang Octava S. Stephani Protomartyr. \gcolor{Simplex.}\\
c &iij & 3 & \hang Octava S. Joannis Apost. et Evang. \gcolor{Simplex.}\\
d &Prid. & 4 & \hang Octava SS. Innocentum Martyrum. \gcolor{Simplex.}\\
e &Non. & 5 & \hang Vigiliæ Epiphianiæ.  \gcolor{Semiduplex.} \mem{S. Telesphori Papæ Martyris.}\\
f &viij & 6 & \hang \capspace{EPIPHANIA DOMINI}. \gcolor{Duplex I classis cum Octava privilegiata II ordinis.}\\
\null & \null & \null & \hang \textit{Dominica infra Octavam Epiphianiæ.} S. Familiæ Jesu, Mariæ, Joseph.  \gcolor{Duplex majus.} \textit{Commem.} Dominicæ et Octavæ.\\
g & vij & 7 & \hang De Octava Epiphianæ. \gcolor{Semiduplex.}\\
\gcolor{A} & vj & 8 & \hang De Octava. \gcolor{Semiduplex.}\\
b & v & 9 &  \hang De Octava. \gcolor{Semiduplex.}\\
c & iv & 10 &  \hang De Octava. \gcolor{Semiduplex.}\\
d & iij & 11 & \hang De Octava. \gcolor{Semiduplex.} \mem{S. Hygini Papæ Martyris.}\\
e & Prid. & 12 &  \hang De Octava.  \gcolor{Semiduplex.}\\
f & Idib. & 13 & \hang Octava Epiphianiæ. \gcolor{Duplex majus.}\\
g & xix & 14 & \hang S. Hilarii Episc. Conf. et Eccl. Doct. \gcolor{Duplex.} \mem{S. Felicis Presbyteri Martyris.}\\
\gcolor{A} & xviij & 15 & \hang S. Pauli primi Eremitæ Conf. \gcolor{Duplex.}\\
b & xvij & 16 & \hang S. Marcelli I Papæ Mart. \gcolor{Semiduplex.}\\
c & xvj & 17 & \hang S. Antonii Abbatis. \gcolor{Duplex.} \mem{S. Mauri Abb.}\\
d & xv & 18 & \hang Cathedra S. Petri Romæ. \gcolor{Duplex majus.} \mem{S. Pauli Ap., ac S. Priscæ Virginis et Martyris.}\\
e & xiv & 19 &  \hang SS. Marii, Marthæ, Audifacis et Abachum Martyrum. \gcolor{Simplex.}\\
f & xiij & 20 & \hang SS. Fabiani Papæ et Sebastiani Martyr. \gcolor{Duplex.}\\
g & xij & 21 & \hang S. Agnetis Virginis et Martyris. \gcolor{Duplex.}\\
\gcolor{A} & xj & 22 & \hang SS. Vincentii et Anastasii Martyrum. \gcolor{Semiduplex.}\\
b & x & 23 & S. Raymundi de Peñafort Conf. \gcolor{Semiduplex.} \mem{S. Emerentianæ Virginis et Martyris.} \\
c & ix & 24 & \hang S. Timothei Episc. Martyris. \gcolor{Duplex.}\\
d & viij & 25 & \hang Conversio S. Pauli Ap. \gcolor{Duplex majus.} \mem{S. Petri Ap.}\\
e & vij & 26 & \hang S. Polycarpi Episc. Martyris. \gcolor{Duplex.}\\
f & vj & 27 & \hang S. Joannis Chrysostomi Episc. Conf. et Eccl. Doct. \gcolor{Duplex.}\\
g & v & 28 & \hang S. Petri Nolasci Conf. \gcolor{Duplex.}\\
\gcolor{A} & iv & 29 & S. Francisci Salesii Episc. Conf. et Eccl. Doct. \gcolor{Duplex.}\\
b & iij & 30 & S. Martinæ Virginis et Martyris.  \gcolor{Semiduplex.}\\
c & Prid. & 31 & \hang S. Joannis Bosco Conf. \gcolor{Duplex.}
\end{longtable}

% !TEX TS-program = LuaLaTeX+se
% !TEX root = Kalendarium.tex

{\centering{{\normalsize \gcolor{Februarius.}}\par}}
\begin{longtable}{>{\centering}p{0.025\textwidth}|>{\raggedright}p{0.040\textwidth}|>{\raggedleft}p{0.025\textwidth}|>{\raggedright\arraybackslash}p{0.80\textwidth}}

d & Kal. & 1 & S. Ignatii Episc. Martyr. \gcolor{Duplex.}\\
e & iv & 2 & \hang \scspace{Purificatio B}. \scspace{Mariæ Virginis}. \gcolor{Duplex II classis.}\\
f & iij & 3 & \hang S. Blasii Episc. Martyr. \gcolor{Simplex.}\\
g & Prid. & 4 & \hang S. Andreæ Corsini Episc. Conf. \gcolor{Duplex.}\\
\gcolor{A} & Non. & 5 & \hang S. Agathæ Virginis et Martyris. \gcolor{Duplex.}\\
b & viij & 6 & \hang S. Titi Episc. Conf. \gcolor{Duplex.} \mem{S. Dorotheæ Virginis et Martyris.} \\
c & vij & 7 & \hang S. Romualdi Abbatis. \gcolor{Duplex.}\\
d & vj & 8 & \hang S. Joannis de Matha Conf. \gcolor{Duplex.}\\
e & v & 9 & S. Cyrilli Episc. Alexandrini Conf. et Eccl. Doct. \gcolor{Duplex.}\\
f & iv & 10 & \hang S. Scholasticæ Virginis. \gcolor{Duplex.}\\
g & iij & 11 & \hang Apparitionis B.M.V. Immaculatæ. \gcolor{Duplex majus.}\\
\gcolor{A} & Prid. & 12 & SS. Septem Fundatorum Ord. Servorum B.V.M., Cc. \gcolor{Duplex.}\\
b & Ibid. & 13 & \\
c & xvj & 14 & \hang S. Valentini Presbyteri Martyris. \gcolor{Simplex.}\\
d & xv & 15 & SS. Faustini et Jovitæ Martyrum. \gcolor{Simplex.}\\
e & xiv & 16 & \\
f & xiij & 17 & \\
g & xij & 18 & S. Simeonis Episcopi Martyris. \gcolor{Simplex.}\\
\gcolor{A} & xj & 19 & \\
b & x & 20 & \\
c & ix & 21 & \\
d & viij & 22 & \hang Cathedra S. Petri Antiochæ. \gcolor{Dupl. maj.} \mem{S. Pauli Ap.}\\
e & vij & 23 & \hang S. Petri Damiani Episc. Conf. et Eccl. Doct. \gcolor{Duplex.} \mem{Vigiliæ.}\\
f & vj & 24 & \scspace{S}. \scspace{Matthiæ Apostoli}. \gcolor{Duplex II classis.}\\
g & v & 25 & \\
\gcolor{A} & iv & 26 & \\
b & iij & 27 & \hang S. Gabrielis a Virgine perdolente Conf. \gcolor{Duplex.}\\
c & Prid. & 28 &\\
& & & \rubrique{In anno bisextili mensis Febrarius est dierum 29, et Festum S. Matthiæ celebratur die 25 Februarii, ac Festum S. Gabrielis a Virg. perdolente die 28 Februarii, et bis dicitur Sexto Kalendas, id est die 24 et die 25; et littera Dominicalis, quæ assumpta fuit in mense Januario, mutatur in præcedentem ut, si in Januario littera Dominicalis fuerit A, mutetur in præcedentum, quæ est \normaltext{g,} etc., littera  \normaltext{f} bis servit 24 et 25.}
\end{longtable}

%In anno bisextili mensis Febrarius est dierum 29, et Festum S. Matthiæ celebratur die 25 Februarii, ac Festum S. Gabrielis a Virg. perdolente die 28 Februarii, et bis dicitur Sexto Kalendas, id est die 24 et die 25; et littera Dominicalis, quæ assumpta fuit in mense Januario, mutatur in præcedentem ut, si in Januario littera Dominicalis fuerit A, mutetur in præcedentum, quæ est g, etc., littera f bis servit 24 et 25. %% contains ad hoc vspace command which could be removed, modified, or replaced with a custom spacing command to avoid magic numbers. rubrique macro is defined in commonheaders file
%% In the current configuration this is commented out and the text is inserted in the calendar table itself for reasons of space. Result TBD as the table of moveable feasts needs serious work and may have to cover two pages.

% !TEX TS-program = LuaLaTeX+se
% !TEX root = Kalendarium.tex

{\centering{{\normalsize \gcolor{Martius.}}\par}}
%
\begin{longtable}{>{\centering}p{0.025\textwidth}|>{\raggedright}p{0.040\textwidth}|>{\raggedleft}p{0.025\textwidth}|>{\raggedright\arraybackslash}p{0.80\textwidth}}
d & Kal. & 1 & \\
e & vj & 2 & \\
f & v & 3 & \\
g & iv & 4 & \hang S. Casimiri Conf. \gcolor{Semid.} \mem{Comm. S. Lucii I Papæ Mart.}\\
\gcolor{A} & iij & 5 & \\
b & Prid. & 6 & SS. Perpetuæ et Felicitatis Martyrum. \gcolor{Duplex.}\\
c & Non. & 7 & \hang S. Thomæ de Aquino Conf. et Eccl. Doct. \gcolor{Duplex.}\\
d & viij & 8 & \hang S. Joannis de Deo Conf. \gcolor{Duplex.}\\
e & vij & 9 & \hang S. Franciscæ Romanæ, Viduæ. \gcolor{Duplex.}\\
f & vj & 10 & SS. Quadraginta Martyrum. \gcolor{Semiduplex.}\\
g & v & 11 & \\
\gcolor{A} & iv & 12 & S. Gregorii Papæ, Conf. et Eccl. Doctoris. \gcolor{Duplex.}\\
b & iij & 13 & \\
c & Prid. & 14 & \\
d & Idib. & 15 & \\
e & xvij & 16 & \\
f & xvj & 17 & \hang S. Patricii Episc. Conf. \gcolor{Duplex.}\\
g & xv & 18 & \hang S. Cyrilli Ep. Hierosolymitani, Conf. et Eccl. Doct. \gcolor{Duplex.}\\
\gcolor{A} & xiv & 19 & \hang \capspace{S. JOSEPH} Sponsi B.M.V., Conf. \gcolor{Duplex I classis.}\\
b & xiij & 20 & \\
c & xij & 21 & S. Benedicti Abbatis. \gcolor{Duplex majus.}\\
d & xj & 22 & \\
e & x & 23 & \\
f & ix & 24 & S. Gabrielis Archangeli. \gcolor{Duplex.}\\
g & viij & 25 & \hang \capspace{ANNUNTIATIO B}. \capspace{MARIÆ VIRGINIS}. \gcolor{Duplex I classis.}\\
\gcolor{A} & vij & 26 & \\
b & vj & 27 & S. Joannis Damasceni Conf. et Ecclesiæ Doctoris. \gcolor{Duplex.}\\
c & v & 28 & S. Joannis a Capistrano Conf. \gcolor{Semiduplex.}\\
d & iv & 29 & \\
e & iij & 30 & \\
f & Prid. & 31 & \\
 &  &  & \hang \textit{Feria VI post Dom. Passionis.} Septum Dolorum B. Mariæ Virginis. \gcolor{Duplex majus.} \mem{Feriæ.}
\end{longtable}

% !TEX TS-program = LuaLaTeX+se
% !TEX root = Kalendarium.tex

{\centering{{\normalsize \gcolor{Aprilis.}}\par}}

\begin{longtable}{>{\centering}p{0.025\textwidth}|>{\raggedright}p{0.040\textwidth}|>{\raggedleft}p{0.025\textwidth}|>{\raggedright\arraybackslash}p{0.80\textwidth}}
g & Kal. & 1 & \\
\gcolor{A} & iv & 2 & \hang S. Francisci de Paula Conf. \gcolor{Duplex.}\\
b & iij. & 3 & \\
c & Prid. & 4 & \hang S. Isidori Episc. Conf. et Eccl. Doctoris. \gcolor{Duplex.}\\
d & Non. & 5 & \hang S. Vincentii Ferrerii Conf. \gcolor{Duplex.}\\
e & viij & 6 & \\
f & vij & 7 & \\
g & vj & 8 & \\
\gcolor{A} & v & 9 & \\
b & iv & 10 & \\
c & iij &11 & \hang S. Leonis I Papæ, Conf. et Eccl. Doctoris. \gcolor{Duplex.}\\
d & Prid. & 12 & \\
e & Ibid. & 13 & \hang S. Hermenegildi Martyris. \gcolor{Semiduplex.}\\
f & xviij & 14 & \hang S. Justini Mart. \gcolor{Duplex.} \mem{SS. Tiburtii, Valeriani et Maximi Martyrum.} \\
g & xvij & 15 & \\
\gcolor{A} & xvj & 16 & \\
b & xv & 17 & \hang S. Aniceti I, Papæ Martyris. \gcolor{Simplex.}\\
c & xiv & 18 & \\
d & xiij & 19 & \\
e & xij & 20 & \\
f & xj & 21 & \hang S. Anselmi Episc. Conf. et Eccl. Doctoris. \gcolor{Duplex.}\\
g & x & 22 & \hang SS. Soteris et Caji Pontif. Mart. \gcolor{Semiduplex.}\\
\gcolor{A} & ix & 23 & \hang S. Georgii Martyris. \gcolor{Semiduplex.}\\
b & viij & 24 & \hang S. Fidelis a Sigmaringa Martyris.\gcolor{Duplex.}\\
c & vij & 25 & \hang S. \scspace{Marci Evangelistæ}. \gcolor{Duplex II classis.}\\
d & vj & 26 & \hang SS. Cleti et Marcellini Pontif. Martyrum. \gcolor{Semiduplex.}\\
e & v & 27 & \hang S. Petri Canisii Conf. et Eccl. Doct. \gcolor{Duplex.}\\
f & iv & 28 & \hang S. Pauli a Cruce Conf. \gcolor{Dupl.}\\
g & iij & 29 & \hang S. Petri Martyris. \gcolor{Duplex.}\\
\gcolor{A} & Prid. & 30 & \hang S. Catharinæ Senensis Virginis. \gcolor{Duplex.}\\
 &  &  & \hang \textit{Feria IV infra Hebdomadam II post octavam Paschæ.} \capspace{SOLEMNITAS S}. \capspace{JOSEPH}, Sponsi B.M.V., Conf. et Eccl. univers. Patroni. \gcolor{Duplex I classis cum Octava communi.}\\
 &  &  & \hang \textit{Feria IV infra Hebdomadam III post octavam Paschæ.} Octava S. Joseph. \gcolor{Duplex majus.}
\end{longtable}

% !TEX root = Kalendarium.tex
% !TEX TS-program = LuaLaTeX+se

{\centering{{\normalsize \gcolor{Maius.}}\par}}

\begin{longtable}{>{\centering}p{0.025\textwidth}|>{\raggedright}p{0.040\textwidth}|>{\raggedleft}p{0.025\textwidth}|>{\raggedright\arraybackslash}p{0.80\textwidth}}
b & Kal. & 1 & \hang \scspace{Ss}. \scspace{Philippi et Jacobi Apostolorum}. \gcolor{Duplex II classis.}\\
c & vj & 2 & \hang S. Athanasii Episc. Conf. et Eccl. Doctoris. \gcolor{Duplex.}\\
d & v & 3 & \hang \scspace{Inventio S}. \scspace{Crucis}. \gcolor{Dupl. II classis.} \mem{SS. Alexandri I Papæ et Soc. Martyrum, ac S. Juvenalis Episc. Conf.}\\
e & iv & 4 & \hang S. Monicæ Viduæ. \gcolor{Duplex.}\\
f & iij & 5 & S. Pii V Papæ Conf. \gcolor{Duplex.}\\
g & Prid. & 6 & \hang S. Joannis Ap. ante Portam Latinam. \gcolor{Duplex majus.}\\
\gcolor{A} & Non. & 7 & S. Stanislai Episc. Martyris. \gcolor{Duplex.}\\
b & viij & 8 & \hang Apparitio S. Michaelis Archangeli. \gcolor{Duplex majus.}\\
c & vij & 9 & \hang S. Grgeorii Nazianzni Episc. Conf. et Eccl. Doct. \gcolor{Duplex.}\\
d & vj & 10 & \hang S. Antonini Episc. Conf. \gcolor{Duplex.}\\
e & v & 11 & \\
f & iv & 12 & \hang SS. Nerei, Achillei et Domitillæ Virginis, atque Pancratii Martyrum. \gcolor{Semiduplex.}\\
g & iij & 13 & \hang S. Roberti Bellarmino Episc. Conf. et Eccl. Doct. \gcolor{Duplex.}\\
\gcolor{A} & Prid. & 14 & \hang S. Bonifatii Martyris. \gcolor{Simplex.}\\
b & Idib. & 15 & \hang S. Joannis Baptistæ de la Salle Conf. \gcolor{Duplex.}\\
c & xvij & 16 & S. Ubaldi Episc. Conf. \gcolor{Semiduplex.}\\
d & xvj & 17 & S. Paschalis Baylon Conf. \gcolor{Duplex.}\\
e & xv & 18 & \hang S. Venantii Martyris. \gcolor{Duplex.}\\
f & xiv & 19 & \hang S. Petri Cœlestini Papæ Conf. \gcolor{Dupl.} \mem{S. Pudentianæ Virginis.}\\
g & xiij & 20 & \hang S. Bernardini Senensis Conf.\gcolor{Semiduplex.}\\
\gcolor{A} & xij & 21 & \\
b & xj & 22 &  \\
c & x & 23 & \\
d & ix & 24 &\\
e & viij & 25 & \hang S. Gregorii VII Papæ Conf. \gcolor{Duplex.} \mem{S. Urbani I Papæ Martyris.}\\
f & vij & 26 & \hang S. Philippi Nerii Conf. \gcolor{Duplex.}\\
g & vj & 27 & \hang S. Bedæ Venerabilis Conf. et Eccl. Doct. \gcolor{Duplex.} \mem{S. Joannis I Papæ Martyris.}\\
\gcolor{A} & v & 28 & S. Augustini Episc. Conf. \gcolor{Duplex.}\\
b & iv & 29 & \hang S. Mariæ Magdalenæ de Pazzis Virg. \gcolor{Semiduplex.}\\
c & iij & 30 & \hang S. Felicis I Papæ Martyris. \gcolor{Simplex.}\\
d & Prid. & 31 & \hang S. Angelæ Mericiæ Virg. \gcolor{Duplex.} \mem{S. Petronillæ Virg.}\\ 
 &  &  & \hang \gcolor{Vel.} \scspace{B}. \scspace{Mariæ Virginis Reginæ}.  \gcolor{Duplex II classis.}  \mem{S. Petronillæ Virg.}
\end{longtable}

% !TEX root = Kalendarium.tex
% !TEX TS-program = LuaLaTeX+se

{\centering{{\normalsize \gcolor{Junius.}}\par}}

\begin{longtable}{>{\centering}p{0.025\textwidth}|>{\raggedright}p{0.040\textwidth}|>{\raggedleft}p{0.025\textwidth}|>{\raggedright\arraybackslash}p{0.80\textwidth}}
e & Kal. & 1 & \\
 &  &  & \hang \gcolor{Vel.} S. Angelæ Mericiæ Virg. \gcolor{Duplex.}\\
f & iv & 2 & SS. Marcellini, Petri atque Erasmi Martyrum. \hang \gcolor{Simplex.}\\
g & iij & 3 & \hang \\
\gcolor{A} & Prid. & 4 & \hang S. Francisci Caracciolo Conf. \gcolor{Duplex.}\\
b & Non. & 5 & \hang S. Bonifatii Episc. Martyris. \gcolor{Duplex.}\\
c & viij & 6 & \hang S. Norberti Episc. Conf. \gcolor{Duplex.}\\
d & vij & 7 & \\
e & vj & 8 & \\
f & v & 9 & \hang SS. Primi et Feliciani Martyrum. \gcolor{Simplex.}\\
g & iv & 10 & S. Margaritæ Reginæ, Viduæ. \gcolor{Semiduplex.}\\
\gcolor{A} & iij & 11 & \hang S. Barnabæ Apostoli. \gcolor{Duplex majus.}\\
b & Prid. & 12 & \hang S. Joannis a S. Facundo Conf.  \gcolor{Dupl.} \mem{Comm. SS. Basilidis, Cyrini, Naboris et Nazarii Martyrum.}\\
c & Idib. & 13 & \hang S. Antonii de Padua Conf. (et Eccl. Doct.) \gcolor{Duplex.}\\
d & xviij & 14 & S. Basilii Magni Episc. Conf. et Eccl. Doct. \gcolor{Duplex.}\\
e & xvij & 15 & SS. Viti, Modesti atque Crescentiæ Martyrum. \gcolor{Simplex.}\\
f & xvj & 16 & \\
g & xv & 17 & \\
\gcolor{A} & xiv & 18 & S. Ephraem Syri Diac., Conf. et Eccl. Doct. \gcolor{Duplex.}\\
b & xiij & 19 & \hang S. Julianæ de Falconeriis Virginis. \gcolor{Dupl.} \mem{SS. Gervasii et Protasii Martyrum.}\\
c & xij & 20 & S. Silverii Papæ Martyris. \gcolor{Simplex.}\\
d & xj & 21 & \hang S. Aloisii Gonzagæ Conf. \gcolor{Duplex.}\\
e & x & 22 & \hang S. Paulini Episc. Conf. \gcolor{Duplex.}\\
f & ix & 23 & Vigilia.\\
g & viij & 24 & \hang \capspace{NATIVITAS S}. \capspace{JOANNIS BAPTISTÆ}. \gcolor{Duplex I classis cum Octava communi.}\\
\gcolor{A} & vij & 25 & S. Gulielmi Abbatis. \gcolor{Duplex.} \mem{Octavæ.}\\
b & vj & 26 & SS. Joannis et Pauli Martyrum. \mem{Octavæ.}\\
c & v & 27 & \hang De Octava. \gcolor{Semiduplex.}\\
d & iv & 28 & \hang S. Irenæi Episc. et Martyr. \gcolor{Duplex.} \mem{Octavæ et Vigiliæ.}\\
e & iij & 29 & \hang \capspace{SS}. \capspace{PETRI ET PAULI APOSTOLORUM}. \gcolor{Duplex I classis cum Octava communi.}\\
f & Prid. & 30 & \hang Commemoratio S. Pauli Apostoli. \gcolor{Duplex majus.} \mem{S. Petri Apostoli et Octavæ S. Joannis Baptistæ.}
\end{longtable}

%%\pagebreak 

% !TEX TS-program = LuaLaTeX+se
% !TEX root = Kalendarium.tex

{\centering{{\normalsize \gcolor{Julius.}}\par}}

\begin{longtable}{>{\centering}p{0.025\textwidth}|>{\raggedright}p{0.040\textwidth}|>{\raggedleft}p{0.025\textwidth}|>{\raggedright\arraybackslash}p{0.80\textwidth}}
g & Kal. & 1 & \hang \capspace{PRETIOSISSIMI SANGUINIS} D.N.J.C. \gcolor{Duplex I classis.} \mem{diei Octavæ S.~Joannis Baptistæ.}\\  %%~ needed with current settings
\gcolor{A} & vj. & 2 & \scspace{Visitatio} B.M.V. \gcolor{Duplex II classis.} \mem{SS. Processi et Martiniani Martyrum} \\
b & v & 3 & \hang S. Leonis II Papæ et Conf. \gcolor{Semiduplex.} \mem{Octavæ.}\\
c & iv & 4 & \hang De Octava. \gcolor{Semiduplex.}\\
d & iij & 5 & \hang S. Antonii Mariæ Zaccaria Conf. \gcolor{Duplex.}\\
e & Prid. & 6 & \hang Octava SS. Petri et Pauli Apostolorum. \gcolor{Duplex majus.}\\
f & Non. & 7 & \hang SS. Cyrilli et Methodii Episc. Conf. \gcolor{Duplex.}\\
g & viij & 8 & \hang S. Elisabeth Reginæ, Viduæ. \gcolor{Semiduplex.}\\
\gcolor{A} & vij & 9 & \\
b & vj & 10 &  \hang SS. Septem Fratrum Martyrum et SS. Rufinæ et Secundæ Virgimum et Martyrum. \gcolor{Semiduplex.}\\
c & v & 11 & \hang S. Pii I Papæ Martyris. \gcolor{Simplex.}\\
d & iv & 12 & S. Joannis Gualberti Abbatis. \gcolor{Duplex.} \mem{SS. Naboris et Felicis Martyrum.}\\
e & iij & 13 & \hang S. Anacleti Papæ Mart. \gcolor{Semiduplex.}\\
f & Prid. & 14 & \hang  S. Bonaventuræ Eepisc. Conf. et Eccl. Doct. \gcolor{Duplex.}\\
g & Idib. & 15 & \hang S. Henrici Imperatoris, Conf. \gcolor{Semiduplex.}\\
\gcolor{A} & xvij &16 & \hang Commemoratio B. Mariæ Virginis de Monte Carmelo. \gcolor{Duplex majus.}\\
b & xvj & 17 &  \hang S. Alexii Conf. \gcolor{Semiduplex.}\\
c & xv & 18 & \hang S. Camilli de Lellis Conf. \gcolor{Dupl.} \mem{SS. Symphorosæ et septem Filiorum ejus Martyrum.}\\
d & xiv & 19 & \hang S. Vincentii a Paulo Conf. \gcolor{Duplex.}\\
e & xiij & 20 & \hang S. Hieronymi Æmiliani Conf. \gcolor{Dupl.} \mem{S. Margaritæ Virg. et Mart.}\\
f & xij & 21 & \hang S. Praxedis Virginis. \gcolor{Simplex.}\\
g & xj & 22 & \hang S. Mariæ Magdalenæ Pœnitentis. \gcolor{Duplex.}\\
\gcolor{A} & x & 23 & \hang S. Apollinaris Episc. Mart. \gcolor{Dupl.} \mem{Comm. S. Liborii Ep. Conf.}\\
b & ix & 24 & \hang  Vigilia. \mem{S. Christinæ Virginis et Martyris.}\\
c & viij & 25 & \hang S. \scspace{Jacobi Apostoli}. \gcolor{Duplex II classis.} \mem{S. Christophori Martyr.}\\
d & vij & 26 & \hang S. \scspace{Annæ Matris} B.M.V. \gcolor{Duplex II classis.}\\
e & vj & 27 & \hang S. Pantaleonis Martyris. \gcolor{Simplex.}\\
f & v & 28 &  \hang SS. Nazarii et Celis Martyrum, Victoris I Papæ Mart. ac Innocentii I Papæ Conf. \gcolor{Semiduplex.}\\
g & iv & 29 & \hang S. Marthæ Virg. \gcolor{Semid.} \mem{SS. Felicis II Papæ, Simplicii, Faustini et Beatricis Martyrum.}\\
\gcolor{A} & iij & 30 & \hang SS. Abdon et Sennen Martyrum. \gcolor{Simplex.}\\
b & Prid. & 31 & \hang S. Ignatii Conf. \gcolor{Duplex majus.}
\end{longtable}

% !TEX TS-program = LuaLaTeX+se
% !TEX root = Kalendarium.tex

{\centering{{\normalsize \gcolor{Augustus.}}\par}}

\begin{longtable}{>{\centering}p{0.025\textwidth}|>{\raggedright}p{0.040\textwidth}|>{\raggedleft}p{0.025\textwidth}|>{\raggedright\arraybackslash}p{0.80\textwidth}}
c & Kal. & 1 & \hang S. Petri ad Vincula. \gcolor{Duplex majus.} \mem{S. Pauli Ap. ac SS. Machabæorum Martyrum.}\\
d & iv & 2 & \hang S. Alpphonsi Mariæ de Ligorio Episc. Conf. et Eccl. Doct. \gcolor{Duplex.} \mem{S. Stephani I Papæ Martyris.}\\
e & iij & 3 & \hang Inventio S. Stephani Protomartyris. \gcolor{Semiduplex.}\\
f & Prid. & 4 & \hang S. Dominici Conf. \gcolor{Duplex majus.}\\
g & Non. & 5 & \hang Dedicatio S. Mariæ ad Nives. \gcolor{Duplex majus.}\\
\gcolor{A} & viij & 6 & \hang \scspace{Transfiguratio} D.N.J.C. \gcolor{Duplex II classis.} \mem{SS. Xysti II Papæ, Felicissimi et Agapiti Martyrum.}\\
b & vij & 7 & \hang S. Cajetani Conf. \gcolor{Duplex.} \mem{S. Donatii Episc. Mart.}\\
c & vj & 8 & \hang SS. Cyriaci, Largi et Smaragdi Martyrum. \gcolor{Semiduplex.}\\
d & v & 9 & \hang S. Joannis Mariæ Vianney Conf. \gcolor{Duplex.} \mem{Vigiliæ et S. Romani Martyris.}\\
e & iv & 10 & \hang S. \scspace{Laurentii Martyris}. \gcolor{Duplex II classis cum Octava simplici.}\\
f & iij & 11 & \hang SS. Tiburtii et Susannæ Virg., Martyrum. \gcolor{Simplex.}\\
 g & Prid. & 12 & \hang S. Claræ Virginis. \gcolor{Duplex.}\\
\gcolor{A} & Ibid. & 13 & \hang SS. Hippolyti et Cassiani Martyrum. \gcolor{Simplex.}\\
b & xix & 14 & \hang Vigilia. \mem{S. Eusebii Conf.}\\
c & xviij & 15 & \hang \capspace{ASSUMPTIO B}. \capspace{MARIÆ VIRGINIS}. \gcolor{Duplex I classis cum Octava communi.}\\
d & xvij & 16 & \hang S. \scspace{Joachim Patris} B.M.V. \gcolor{Duplex II classis.}\\
e & xvj & 17 & \hang S. Hyacinthi Conf. \gcolor{Duplex.} \mem{Octavæ Assumptionis ac diei Octavæ S. Laurentii Mart.} \\
f & xv & 18 & \hang De Octava Assumptionis. \gcolor{Semiduplex.} \mem{S. Agapiti Mart.}\\
g & xiv & 19 & \hang S. Joannis Eudes. \gcolor{Duplex.} \mem{Octavæ.}\\
\gcolor{A} & xiij & 20 & \hang S. Bernardi Abbatis Conf. et Eccl. Doct. \gcolor{Duplex.} \mem{Octavæ.}\\
b & xij & 21 & \hang S. Joannæ Franciscæ Fremiot de Chantal Viduæ. \gcolor{Duplex.} \mem{Octavæ.}\\
c & xj & 22 & \hang Octava Assumptionis B.M.V. \gcolor{Duplex majus.} \mem{SS Timothei, Hippolyti et Symphoriani Martyrum.}\\
& & & \hang \gcolor{Vel.} \scspace{Immaculati Cordis} B.M.V. \gcolor{Duplex II classis.}  \mem{SS Timothei, Hippolyti et Symphoriani Martyrum.}\\
d & x & 23 & \hang S. Philippi Benitii Conf. \gcolor{Duplex.} \mem{Vigiliæ.}\\
e & ix & 24 & \hang S. \scspace{Bartholomæi Apostoli}. \gcolor{Duplex II classis.}\\
f & vijj & 25 & \hang S. Ludovici Regis, Conf. \gcolor{Semiduplex.}\\
g & vij & 26 &  \hang S. Zephyrini Papæ Martyris. \gcolor{Simplex.}\\
\gcolor{A} & vj & 27 & \hang S. Josephi Calasanctii Conf. \gcolor{Duplex.}\\
b & v & 28 & \hang S. Augustini Episc. Conf. et Eccl. Doct. \gcolor{Dupl.} \mem{S. Hermetis Martyris.}\\
c & iv & 29 & \hang Decollatio S. Joannis Baptistæ. \gcolor{Duplex majus.} \mem{S. Sabinæ Martyris.}\\
d & iij & 30 &  \hang S. Rosæ Limanæ Virginis. \gcolor{Duplex.} \mem{SS. Felicis et Adaucti Martyrum.}\\
e & Prid. & 31 &  \hang S. Raymundi Nonnati Conf. \gcolor{Duplex.}\\
\end{longtable}


% !TEX TS-program = LuaLaTeX+se
% !TEX root = Kalendarium.tex

{\centering{{\normalsize \gcolor{September.}}\par}}

\begin{longtable}{>{\centering}p{0.025\textwidth}|>{\raggedright}p{0.040\textwidth}|>{\raggedleft}p{0.025\textwidth}|>{\raggedright\arraybackslash}p{0.80\textwidth}}
 f & Kal. & 1 & S. Ægidii Abbatis. \gcolor{Simplex.} \mem{SS. Duodecim Fratrum Martyrum.}\\
g & iv & 2 & \hang S. Stephani Hungariæ Regis, Conf. \gcolor{Semiduplex.}\\
\gcolor{A} & iij & 3 & \\
& & & \hang \gcolor{Vel.} S. Pii X Papæ et Conf. \gcolor{Duplex.}\\
b & Prid. & 4 & \\
c & Non. & 5 & S. Laurentii Justiniani Episc. Conf. \gcolor{Duplex.}\\
d & viij & 6 &  \\
e & vij & 7 & \\
f & vj & 8 & \hang \scspace{Nativitas B}. \scspace{Mariæ Virginis}. \gcolor{Duplex II Classis cum Octava simplici.} \mem{S. Hadriani Martyris.}\\
g & v & 9 & \hang S. Gorgonii Martyris. \gcolor{Simplex.}\\
\gcolor{A} & iv & 10 & \hang S. Nicolai a Tolentino. \gcolor{Duplex.}\\
b & iij & 11 & SS. Proti et Hyacinthi Mart.  \gcolor{Simplex.}\\
c & Prid. & 12 & \hang Ss. Nominis Mariæ. \gcolor{Duplex majus.}\\
d & Idib. & 13 & \hang \\
e & xviij & 14 & \hang Exaltatio S. Crucis. \gcolor{Duplex majus.}\\
f & xvij & 15 & \hang \scspace{Septum Dolorum B}. \scspace{Mariæ Virginis}. \gcolor{Duplex II Classis.} \mem{S. Nicomedis Mart.}\\
g & xvj & 16 & \hang SS. Cornelii Papæ et Cypriani Episc., Mart. \gcolor{Semid.} \mem{SS. Euphemiæ et Sociorum Martyrum.}\\
\gcolor{A} & xv & 17 & \hang Impressio sacrorum Stigmatum S. Francisci Conf. \gcolor{Duplex.}\\
b & xiv & 18 & \hang S. Josephi a Cupertino Conf. \gcolor{Duplex.}\\
c & xiij & 19 & \hang SS. Januarii Episcopi et Soc. Martyrum. \gcolor{Duplex.}\\
d & xij & 20 & \hang SS. Eustachii et Soc. Martyrum. \gcolor{Duplex.} \mem{Vigiliæ.}\\
e & xj & 21 & \hang S. \scspace{Matthæi Apostoli et Evangelistæ}. \gcolor{Duplex II classis.}\\
f & x & 22 &  \hang S. Thomæ de Villanova Episc. Conf. \gcolor{Dupl.} \mem{SS. Mauritii et Sociorum Martyrum.}\\
g & ix & 23 & \hang S. Lini Papæ Mart. \gcolor{Semid.} \mem{S. Theclæ Virg. et Mart.}\\
\gcolor{A} & viij & 24 &  \hang B. Mariæ Virginis de Mercede. \gcolor{Duplex majus.}\\
b & vij & 25 &  \\
c & vj & 26 & \hang SS. Cypriani et Justinæ Virginis, Martyrum. \gcolor{Simplex.}\\
d & v & 27 & \hang SS. Cosmæ et Damiani Martyrum. \gcolor{Semiduplex.}\\
e & iv & 28 & \hang S. Wenceslai Ducis, Martyris. \gcolor{Semiduplex.}\\
f & iij & 29 & \hang \capspace{DEDICATIO S}. \capspace{MICHAELIS ARCHANGELI}. \gcolor{Dupl. I classis.}\\
g & Prid. & 30 & \hang S. Hieronymi Presbyteri, Conf. et Eccl. Doctoris. \gcolor{Duplex.}
\end{longtable}


% !TEX root = Kalendarium.tex

{\centering{{\normalsize \gcolor{October.}}\par}}

\begin{longtable}{>{\centering}p{0.025\textwidth}|>{\raggedright}p{0.040\textwidth}|>{\raggedleft}p{0.025\textwidth}|>{\raggedright\arraybackslash}p{0.80\textwidth}}
\gcolor{A} & Kal. & 1 & \hang S. Remigii Episc. Conf. \gcolor{Simplex.}\\
b & vj & 2 & \hang SS. Angelorum Custodum. \gcolor{Duplex majus.}\\
c & v &3 & S. Teresiæ a Jesu Infante Virg. \gcolor{Duplex.}\\
d & iv & 4 & \hang  S. Francisci Confessoris. \gcolor{Duplex majus.}\\
e & iij & 5 & \hang SS. Placidi et Sociorum Martyrum. \gcolor{Simplex.}\\
f & Prid. & 6 & \hang S. Brunonis Confessoris. \gcolor{Duplex.}\\
g & Non. & 7 & \hang \scspace{Sacratissimi Rosarii} B.M.V. \gcolor{Duplex II Classis.} \mem{S. Marci Papæ Conf. ac SS. Sergii et Sociorum Mart.}\\
\gcolor{A} & viij & 8 & S. Birgittæ Viduæ. \gcolor{Duplex.}\\
b & vij & 9 & \hang S. Joannis Leonardi Conf. \gcolor{Duplex.} \mem{SS. Dionysii, Episc., Rustici et Eleutherii Mart.}\\
c & vj & 10 & S. Francisci Borgiæ Confessoris. \gcolor{Semiduplex.}\\
d & v & 11 & \hang \scspace{Maternitatis B}. \scspace{Mariæ Virginis.} \gcolor{Duplex II Classis.}\\
e & iv & 12 & \\
f & iij & 13 & S. Eduardi Regis, Conf. \gcolor{Semiduplex.}\\
g & Prid. & 14 & \hang S. Callisti I Papæ Martyris. \gcolor{Duplex.}\\
\gcolor{A} & Idib. & 15 & \hang S. Teresiæ Virginis. \gcolor{Duplex.}\\
b & xvij & 16 & \hang S. Hedwigis Viduæ. \gcolor{Semiduplex.}\\
c & xvj & 17 & \hang  S. Margaritæ Mariæ Alacoque. \gcolor{Duplex.}\\
d & xv & 18 & \hang S. \scspace{Lucæ Evanglistæ}. \gcolor{Duplex II Classis.}\\
e & xiv &19 & \hang S. Petri de Alcantara Conf. \gcolor{Duplex.}\\
f & xiij & 20 & \hang S. Joannis Cantii Conf. \gcolor{Duplex.}\\
g & xij & 21 & \hang S. Hilarionis Abbatis. \gcolor{Simplex.} \mem{SS. Ursulæ ac Sociarum Virginum et Martyrum.}\\
\gcolor{A} & xj & 22 & \\
b & x & 23 & \hang \\
c & ix & 24 & \hang S. Raphaelis Archangelis. \gcolor{Duplex majus.}\\
d & viij & 25 & \hang SS. Chrystani et Dariæ Martyrum. \gcolor{Simplex.}\\
e & vij & 26 & \hang S. Evaristi Papæ Martyris. \gcolor{Simplex.}\\
f & vj & 27 & \hang Vigilia.\\
g & v & 28 & \hang \scspace{Ss}. \scspace{Simonis et Judæ Apostolorum}. \gcolor{Duplex II Classis.}\\
\gcolor{A} & iv & 29 & \\
b & iij & 30 & \\
c & Prid. & 31 & \hang Vigilia Omnium Sanctorum.\\
& & & \hang \textit{Dominica ultima Octobris.} \capspace{FESTUM} D.N. \capspace{JESU CHRISTI REGIS}. \gcolor{Duplex I classis.}
\end{longtable}


% !TEX root = Kalendarium.tex

{\centering{{\normalsize \gcolor{November.}}\par}}

\begin{longtable}{>{\centering}p{0.025\textwidth}|>{\raggedright}p{0.040\textwidth}|>{\raggedleft}p{0.025\textwidth}|>{\raggedright\arraybackslash}p{0.80\textwidth}}
d & Kal. & 1 & \hang \capspace{OMNIUM SANCTORUM}. \gcolor{Duplex I classis cum Octava communi.}\\
e & iv & 2 & \hang Commemoratio Omnium Fidelium Defunctorum. \gcolor{Duplex.}\\
f & iij & 3 & \hang De Octava Omnium Sanctorum. \gcolor{Semiduplex.} \mem{Octavæ ac SS. Vitalis et Agricolæ Martyrum.}\\
g & Prid. & 4 & \hang S. Caroli Episc. Conf. \gcolor{Duplex.}\\
\gcolor{A} & Non. & 5 & \hang De Octava. \gcolor{Semiduplex.}\\
b & viij & 6 & \hang De Octava. \gcolor{Semiduplex.}\\
c & vij & 7 & \hang  De Octava. \gcolor{Semiduplex.}\\
d & vj & 8 & \hang Octava Omnium Sanctorum.  \gcolor{Duplex majus.} \mem{SS. Quatuor Coronatorum Martyrum.}\\
e & v & 9 & \hang \scspace{Dedicatio Archibasilicæ Ss. Salvatoris}. \gcolor{Duplex II classis.} \mem{S.~Theodori Martyris.}\\ %%~ needed with current settings
f & iv & 10 & \hang  S. Andreæ Avellini Conf. \gcolor{Duplex.} \mem{SS. Tryphonis et Sociorum Martyrum.}\\
g & iij & 11 & \hang  S. Martini Episc. Conf. \gcolor{Duplex.} \mem{S. Mennæ Mart.}\\
\gcolor{A} & Prid. & 12 & \hang S. Martini I Papæ Martyris. \gcolor{Semiduplex.}\\
b & Idib. & 13 & S. Didaci Conf. \gcolor{Semiuplex.}\\
c & xviij & 14 & S. Josaphat Episc. Martyris. \gcolor{Duplex.}\\
d & xvij & 15 & \hang S. Alberti Magni Ep., Conf. et Eccl. Doct. \gcolor{Duplex.}\\
e & xvj & 16 & \hang S. Gertrudis Virginis. \gcolor{Duplex.}\\
f & xv &17 & \hang S. Gregorii Thaumaturgi Episc. Conf. \gcolor{Semiduplex.}\\
g & xiv & 18 & \hang Dedicatio Basilicarum SS. Petri et Pauli Apost. \gcolor{Dupl. majus.}\\
\gcolor{A} & xiij & 19 & \hang S. Elisabeth Viduæ. \gcolor{Dupl.} \mem{S. Pontiani Papæ Mart.}\\
b & xij & 20 & \hang S. Felicis de Valois Conf. \gcolor{Duplex.}\\
c & xj & 21 & \hang Præsentatio B. Mariæ Virginis. \gcolor{Duplex majus.}\\
d & x & 22 & \hang S. Cæciliæ Virginis et Martyris. \gcolor{Duplex.}\\
e & ix & 23 & \hang S. Clementis I Papæ Martyris. \gcolor{Duplex.} \mem{S. Felicitatis Martyris.}\\
f & viij & 24 & \hang S. Joannis a Cruce Conf. et Eccl. Doct. \gcolor{Duplex.} \mem{S. Chrysogoni Mart.}\\
g & vij & 25 & \hang S. Catharinæ Virginis et Martyris. \gcolor{Duplex.}\\
\gcolor{A} & vj & 26 & \hang S. Silvestri Abbatis. \gcolor{Duplex.} \mem{S. Petri Alexandrini Episc. Martyris.}\\
b & v & 27 & \\
c & iv & 28 & \\
d & iij & 29 & Vigilia. \mem{S. Saturnini Martyris.}\\
e & Prid. & 30 & \hang \scspace{S}. \scspace{Andreæ Apostoli}. \gcolor{Duplex II classis.}
\end{longtable}

% !TEX root = Kalendarium.tex

{\centering{{\normalsize \gcolor{December}}\par}}

\begin{longtable}{>{\centering}p{0.025\textwidth}|>{\raggedright}p{0.040\textwidth}|>{\raggedleft}p{0.025\textwidth}|>{\raggedright\arraybackslash}p{0.80\textwidth}}
f & Kal. & 1 & \\
g & iv & 2 & S. Bibianæ Virginis et Martyris. \gcolor{Semiduplex.}\\
\gcolor{A} & iij & 3 & \hang S. Francisci Xaverii Conf. \gcolor{Duplex majus.}\\
b & Prid. & 4 & \hang S. Petri Chrysologi Episc. Conf. et Eccl. Doctoris. \gcolor{Duplex.} \mem{S. Barbaræ Virginis et Martyris.}\\
c & Non. & 5 & \hang \mem{S. Sabbæ Abbatis.}\\
d & viij & 6 & \hang S. Nicolai Episc. Conf. \gcolor{Duplex.}\\
e & vij & 7 & \hang S. Ambrosii Episc. Conf. et Eccl. Doct. \gcolor{Duplex.}\\ %% the sources disagree: LA1949 has no mention of the Vigil. AM1934 mentions it but has a note "de qua nihil in Officio" AR 1912 has (Vigilia.)
f & vj & 8 & \hang \capspace{CONCEPTIO IMMACULATA B}. \capspace{MARIÆ VIRGINIS}. \gcolor{Duplex I classis cum Octava communi.}\\ 
g & v & 9 & \hang De Octava Conceptionis. \gcolor{Semiduplex.}\\
\gcolor{A} & iv & 10 & \hang De Octava. \gcolor{Semiduplex.} \mem{S. Melchiadis Papæ Martyr.}\\
b & iij & 11 & \hang S. Damasi I Papæ Conf. \gcolor{Semiduplex.} \mem{Octavæ.}\\
c & Prid. & 12 & \hang De Octava. \gcolor{Semiduplex.}\\
d & Idib. & 13 & \hang S. Luciæ Virginis et Martyris. \gcolor{Duplex.} \mem{Octavæ.}\\
e & xix & 14 & \hang De Octava. \gcolor{Semiduplex.}\\
f & xviij & 15 & Octava Conceptionis Immaculatæ B.M.V. \gcolor{Duplex majus.}\\
g & xvij & 16 & S. Eusebii Episc. Martyris.  \gcolor{Semiduplex.}\\
\gcolor{A} & xvj & 17 & \\
b & xv & 18 & \\
c & xiv & 19 & \\
d & xiij & 20 & Vigilia.\\
e & xij & 21 & \hang S. \scspace{Thomæ Apostoli}. \gcolor{Duplex I classis.}\\
f & xj & 22 & \\
g & x & 23 & \\
\gcolor{A} & ix & 24 & Vigilia.\\
b & viij & 25 & \hang \capspace{NATIVITAS} D.N. \capspace{JESU CHRISTI}. \gcolor{Duplex I classis cum Octava privilegiata III ordinis.}\\
c & vij & 26 & \hang S. \scspace{Stephani Protomartyris}. \gcolor{Duplex II classis cum Octava simplici.} \mem{Octavæ Nativitatis.}\\ 
d & vj & 27 & \hang S. \scspace{Joannis Apostoli et Evangelistæ}. \gcolor{Duplex II classis cum Octava simplici.} \mem{Octavæ Nativitatis.}\\
e & v & 28 & \hang \scspace{Ss}. \scspace{Innocentium Martyrum} \gcolor{Duplex II classis cum Octava simplici.} \mem{Octavæ Nativitatis.}\\
f & iv & 29 & \hang S. Thomæ Episcopi Martyris.\gcolor{Duplex.}\\
g & iij & 30 & De Octava Nativitatis.  \gcolor{Semiduplex.}\\
\gcolor{A} & Prid. & 31 & \hang S. Silvestri I Papæ Conf. \gcolor{Duplex.} \mem{Comm. Octavæ Nativitatis.}
\end{longtable}


\normalsize

\end{document}