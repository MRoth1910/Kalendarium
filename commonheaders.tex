%%%%%%%%%%%%%%% STANDARD PACKAGES %%%%%%%%%%%%%%%

%% for personal reasons, the commands are a mix of French and English names. Feel free to choose one or the other if you make use of them. This file is subject to change later on.

%% This is the format of the recent Solesmes books.
%\usepackage[paperwidth=6in, paperheight=9in]{geometry}
\usepackage[paperheight=205mm,paperwidth=135mm]{geometry}
\usepackage{fontspec}
\usepackage[ecclesiasticlatin.usej]{babel}
     \babelprovide[hyphenrules=latin]{ecclesiasticlatin}
          \usepackage{xspace}
%     \usepackage{multicol}
%\usepackage{paracol}
%\footnotelayout{m} %% this allows merged footnote if text is in parallel columns (mostly translations)
\usepackage{fancyhdr}
\usepackage{needspace} %%pour réserver de l'espace en bas de page ; ça évite des séparations entre les titres et la partition ou le texte qui les suivent.
\usepackage[compact]{titlesec}
\usepackage{xcolor}
%\usepackage{xstring}
\usepackage{enumitem} %% nécessaire pour les textes des psaumes.
\usepackage{hyperref}
\usepackage{refcount}


%%%%%%%%%%%%%%% HYPHENATION AND TYPOGRAPHICAL CONVENTIONS %%%%%%%%%%%%%%

%% page settings %%%%
\geometry{bindingoffset=5mm,
inner=10mm,
outer=10mm,
top=12mm,
bottom=15mm,
headsep=3mm,
} %%borrowed from Matthias, based on the Solesmes books; binding offset TBD.

\AtBeginDocument{\setlength{\parindent}{1em}} %% should set 1em to EB Garamond not Computer Modern.

%%%%%%%%%%%%%%% GREGORIO CONFIG %%%%%%%%%%%%%%%

\usepackage[autocompile]{gregoriotex}

%% sets * and † to font called via fontspec
\def\GreStar{*}
\def\GreDagger{†}
%\gresetspecial{cross}{\grecross} %% to use cross from Gregorio fonts which is thin and doesn't especially match text well.

%produces a score with a smaller initial (default is 40 pt) and annotations like in the Solesmes books; the 1st is optional and can be omitted as needed.
 \NewDocumentCommand{\gscore}{O{}mm}{%
 \greannotation{#1}%
\greannotation{#2}%
\grechangestyle{initial}{\fontsize{28}{28}\selectfont}%
 \gregorioscore{partitions/#3}}

%% score with no initial, e.g psalms, common tones, etc.
\newcommand{\smallscore}[2][y]{
  \gresetinitiallines{0}
  \gregorioscore{partitions/#2} %% à remplacer avec NewDocumentCommand ; les arguments ne marchent pas correctement … que fait le [y] ?
  \gresetinitiallines{1}
}

 \NewDocumentCommand{\MagAnnotation}{}{%
 \grechangedim{annotationseparation}{-0.01cm}{fixed}%
 } %% pour les antiennes où le chiffre convient mieux aux miniscules (parfois 1, 2, 4, 7…) ; il n'est pas nécessaire pour 6 ou 8.
%\gresetheadercapture{annotation}{greannotation}{string}

%% text above lines shall be italicized and in a smaller font
\grechangestyle{abovelinestext}{\it\fontsize{8}{10}\selectfont}

%% We do not want alt (A-bove l-ines t-ext) to be it for psalm and canticle endings,, however;  Roman/upright text is what the Liber Usualis uses.

%% fine-tuning of space beween the staff and the text above lines (used for Magnificats; needs to be rethought (again, should 2 and 3 be raised or not?) and worked into \smallscore
\newcommand{\altraise}{-0.2cm} %% default is -0.1cm %% needs to be fine-tuned for \rubrique command with EB Garamond font. -0.2cm is MB value
\grechangedim{abovelinestextraise}{\altraise}{scalable}

\newcommand{\altheight}{0.5cm} %% default is 0.3cm and value must be bigger than text height; this should fix the problem. but needs further investigation.
\grechangedim{abovelinestextheight}{\altheight}{scalable} %% this is a very finicky command.

%% allows printing of glyphs from Gregorio score font (available glyphs listed in documentation) as text.
\makeatletter
\def\gretextglyph#1{{\gre@font@music\csname GreCP#1\endcsname}}
\makeatother

%%% Font %%%
\setmainfont{EB Garamond}[UprightFont=EB Garamond Regular,
ItalicFont= EB Garamond Italic,
BoldFont= EB Garamond Bold,
Ligatures=Rare,
Numbers=Proportional,
Numbers=OldStyle] %% all \kern numbers are based on this and can be removed if this font is abandoned.
\newfontfamily\symbolfont{liturgy}
  %% text must be written {\symbolfont{+}} (you can use V and R as well) but may conflict as + is † in gabc. Cross should be \small or even \footnotesize
%\newfontfamily\symbolfont2{BorgiaPro-Regular}

%%%% Font %%%
%\setmainfont{EB Garamond}[UprightFont=EB Garamond Regular,
%ItalicFont= EB Garamond Italic,
%BoldFont= EB Garamond Bold,
%Ligatures=Rare,
%StylisticSet=6,
%Numbers=Proportional,
%Numbers=OldStyle] %% all \kern numbers are based on this and can be removed if this font is abandoned. %%StylisticSet=6 is, in Pardo EBGaramond, the long-tailed Q.
%\newfontfamily\symbolfont{liturgy} 
  %% text must be written {\symbolfont{+}} (you can use V and R as well) but may conflict as + is † in gabc. Cross should be \small or even \footnotesize


%%% Typographical Commands %%%%

%%rubrics: black italics, smaller than body of psalms etc

\NewDocumentCommand{\rubrique}{m}{{\fontsize{8}{10}\selectfont\textit{#1}}}

%\NewDocumentCommand{\rubrique}{m}{{\scriptsize\textit{#1}}}

%% macro to print Alleluia for versicles.
\NewDocumentCommand{\tpalleluia}{}{(\textit{T.P.} Allelúia.)}

%% macro to print normal text inside of rubric (name of a chant or prayer, etc.)

\NewDocumentCommand{\normaltext}{m}{{\normalfont\fontsize{8}{10}\selectfont{#1}}}

%% in case something should be bolded inside of a rubrique
\NewDocumentCommand{\rubriquegras}{m}{{\fontsize{8}{10}\selectfont\textbf{#1}}}

%% to print in red instead of italicizing
\NewDocumentCommand{\rouge}{m}{\textcolor{gregoriocolor}{\fontsize{8}{10}\selectfont{#1}}}

%% to print in black within rouge group.
\NewDocumentCommand{\textenoir}{m}{\textcolor{black}{\fontsize{8}{10}\selectfont\normalfont{#1}}}

%% rouge but in italic (this is technically not correct).
\NewDocumentCommand{\rougeit}{m}{\textcolor{gregoriocolor}{\fontsize{8}{10}\selectfont\textit{#1}}}

%\newcommand{\rubriquegras}[1]{{\fontsize{9}{11}\selectfont\textbf{#1}}}

%%macro to space punctuation with ecclesiasticlatin language from babel.
\NewDocumentCommand{\espaceponctuation}{}{\hspace{0.10em}} %%this value seems more balanced with this font than the definition provided in the ecclesiasticlatin documentation.

\NewDocumentCommand{\myrule}{}{%
    \par%
    {%
        \centering%
        \rule{0.3\textwidth}{0.4pt}%
        \par%
    }% typesets a horizontal rule like on the page with the prayers before and after the office.
}  %%If you're using color at all in your document, you might want to either force \myrule to use black or make it cusotmizable. (from u/Independent-Comb-257)


%%% typesets a cross pattée.

\NewDocumentCommand{\mycross}{}{%
{\symbolfont\footnotesize{{+}}}%
{}
} %% can replace <+> with <U+2720> if a suitable font is found (or EB Garamond font is fixed); command will be useful in lieu of typing the Unicode.

%% macro to format psalm text

\NewDocumentCommand{\pstexte}{ m }{%
    \smallskip%
    \noindent%
    \begin{itemize}[%
    		label=\null, %
			leftmargin=0pt, %
			itemindent=0mm, %
			labelsep=0pt, %
			labelwidth=0pt, %
			rightmargin=0pt, %
			parsep=0pt, %
			topsep=0pt, %
			itemsep=0pt]%
    \input{psaumes/#1}
    \end{itemize}}
    
    %% Command de Matthias Bry modifié, prints psalm incipit score with the text
    \NewDocumentCommand{\psalmus}{mmm}{
	\needspace{4\baselineskip}
	\smalltitle{Psalmus #1.}
	\smallscore[n]{#2}
	\pstexte{#3}
}
    
    %% macro to print any additional text (Capitiulum, oratio, rubrics)
 \NewDocumentCommand{\textes}{ m }{%
    \input{textes/#1}}
    
    
    %% SC work for page headers and for secondary headings but not so much for things like "Dominica…" and certainly not the feast name%%
    
    %%%% Headers%%%%
    \pagestyle{fancy}
\fancyhead{}
\fancyfoot{}
\renewcommand{\headrulewidth}{0pt}


%\fancyhead[RO]{\small\thepage}
%\fancyhead[LE]{\small\thepage}

\fancyhead[RO]{\small\rightmark\hspace{0.5cm}\thepage}
\fancyhead[LE]{\small\thepage\hspace{0.5cm}\leftmark}

\newcommand{\setheaders}[2]{
	\renewcommand{\rightmark}{{\sc#2}}
	\renewcommand{\leftmark}{{\sc#1}}
}
\setheaders{}{}


%% TITLE Commands %%%%
\titleformat{\section}[display]{\large\filcenter\sc\addfontfeature{LetterSpace=3.0}}{}{}{}
\titleformat{\subsection}[display]{\large\filcenter\sc\addfontfeature{LetterSpace=3.0}}{}{}{}
\setcounter{secnumdepth}{0}
\titlespacing{\section}{0pt}{*0}{*0}

%% all of these are a mess and should probably be ignored, but feel free to use them by referring to Matthias B's originals in the Nocturnale Romanum repository.

\newcommand{\officiumtitulum}[1]{
%  \newpage
%  \thispagestyle{empty}
  \setheaders{{\scshape\addfontfeature{LetterSpace=3.0}#1}}{{\scshape\addfontfeature{LetterSpace=3.0} #1}}
  \begin{center}
  {\scshape\large #1}\par
  \end{center}}
  
%   \setheaders{{{\scshape\addfontfeature{LetterSpace=3.0}}}{{{\scshape\addfontfeature{LetterSpace=3.0} {#1}}}
\NewDocumentCommand{\smalltitle}{m}{
\needspace{5\baselineskip}  %% very small if there are neumes above the staff, including flats in mode 2, e.g. O Doctor optime
\vspace{\baselineskip}
 {\centering #1\par}
}

\newcommand{\subtitulum}[1]{
 \subsection{
  \begin{center}
  {\large#1}
  \end{center}
}}

%% originally used apparently for typesetting portions of the Common of the BVM
\newcommand{\espacetitre}{\vspace{-0.5cm}}
\newcommand{\espaceps}{\vspace{-3mm}} %% ps =psaume
\newcommand{\espacecap}{\vspace{-3mm}} %% cap is capitulum
\newcommand{\espace}{\vspace{2mm}}


